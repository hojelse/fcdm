\documentclass[a4paper,11pt]{article}

% Import packages
\usepackage[a4paper]{geometry}
\usepackage[utf8]{inputenc}
\usepackage{amsmath}
\usepackage{amssymb}
\usepackage{bussproofs}
\usepackage{proof}
\usepackage{tikz}

\newcommand*{\QEDA}{\hfill\ensuremath{\blacksquare}}%
\newcommand*{\QEDB}{\hfill\ensuremath{\square}}%

% Change enumerate environments you use letters
\renewcommand{\theenumi}{\alph{enumi}}

\title{Peergrade assignment 3}
\author{Author: Kristoffer Højelse} 
\date{\today}

\begin{document} 

\maketitle

\section*{Exercise 1}

\subsection*{1.1.}

\noindent (a) \textbf{true}

Example: $a,d$ is a walk without repeated edges and vertices.

\noindent (b) \textbf{false}

Counterexample.

$c$ and $d$ is not connected.

Therefore there is no simple path from $c$ to $d$.

\noindent (c) \textbf{false}

Counterexample.

There is no path from $c$ to $d$
\subsection*{1.2.}

\noindent (a) \textbf{true}

The given graph is a tree, because the graph is connected and there doesn't exist a simple circuit.

\noindent (b) \textbf{true}

There exists a way to root the tree such that the graph is a binary tree.

\begin{center}
\begin{tikzpicture}[xscale=0.6,yscale = 1]
    \node (a) at (0,5) {$a$};
    \node (b) at (-1,4) {$b$};
    \node (c) at (-4,1) {$c$};
    \node (d) at (-3,2) {$d$};
    \node (e) at (1,4) {$e$};
    \node (f) at (-2,1) {$f$};
    \node (g) at (-2,3) {$g$};
    \node (h) at (0,3) {$h$};
    \draw (a) -- (b) -- (g) -- (d) -- (c);
    \draw (a) -- (e);
    \draw (b) -- (h);
    \draw (d) -- (f);
\end{tikzpicture}
\end{center}

If you root the tree by vertex b or d, the graph will not be a binary tree.

\noindent (c) \textbf{true}

There exists a way to root and arrange the graph such that the graph is a complete binary tree.

\begin{center}
\begin{tikzpicture}[xscale=0.6,yscale = 1]
    \node (a) at (-3,-2) {$a$};
    \node (b) at (-2,-1) {$b$};
    \node (c) at (1,-2) {$c$};
    \node (d) at (2,-1) {$d$};
    \node (e) at (-3,-3) {$e$};
    \node (f) at (3,-2) {$f$};
    \node (g) at (0,0) {$g$};
    \node (h) at (-1,-2) {$h$};
    \draw (g) -- (b) -- (a) -- (e);
    \draw (b) -- (h);
    \draw (g) -- (d) -- (f);
    \draw (d) -- (c);
\end{tikzpicture}
\end{center}

\newpage
\section*{Exercise 2}

\subsection*{2.1.}
(a)

$a_{[2;9]} = 2^2, 3^2, 4^2, 5^2, 6^2, 7^2, 8^2, 9^2$

$a_{[2;9]} = 4,9,16,25,36,59,64,81$

\noindent (b)

$b_{[2;9]} = {1 \over 2}, {1 \over 3}, {1 \over 4}, {1 \over 5}, {1 \over 6}, {1 \over 7}, {1 \over 8}, {1 \over 9}$

\noindent (c)

$c_2 = a_1 \cdot b_1 + a_2 \cdot b_2 = 1 \cdot {1 \over 1} + 4 \cdot {1 \over 2} = 1 + 2 = 3$

apparently $k$ can be described as $k = a_k \cdot b_k$ therefore $c_k$ can be described as:

$c_k = c_{k-1} + k$ for $k > 2$

$c_3 = c_2 + 3 = 3 + 3 = 6$

$c_4 = 6 + 4 = 10$

$c_5 = 10 + 5 = 15$

$c_6 = 15 + 6 = 21$

$c_7 = 21 + 7 = 28$

$c_8 = 28 + 8 = 36$

$c_9 = 36 + 9 = 45$

$c_{[2;9]} = 3,10,15,21,28,36,45$

\subsection*{2.2.}
(a)

$(763636 \cdot 437813 \cdot 936257) \mod 43$

$= ((763636 \mod 43) \cdot (437813 \mod 43) \cdot (936257 \mod 43)) \mod 43$

$= 19$
\noindent (b)

$(763636 \cdot (437813 + 936257)) \mod 43$

$= ((894461 \mod 59) \cdot ((206193 \mod 59) + (83218 \mod 59))) \mod 59$

$= 41$

\subsection*{2.3.}

(a)\\
\noindent Octal to binary

Each character in octal can be represented by exactly 3 bits. 

Therefore I'm just concatenating blocks of 3 bits starting with the least significant digits.
Example: 
$(7)_8 = (111)_2$ and $(1)_8 = (001)_2$ therefore  $(17)_8 = (001111)_2$.\\

$(73217)_8 = (111011010001111)_2$\\

\noindent Octal to decimal

$(73217)_8 = (7 \cdot 8^0 + 8 \cdot 8^1 + 2 \cdot 8^2 + 3 \cdot 8^3 + 7 \cdot 8^4)_{10} = (30351)_{10}$

\noindent (b)

$62290 = 8 \cdot 7786 + 2$

$7786 = 8 \cdot 973 + 2$

$973 = 8 \cdot 121 + 5$

$121 = 8 \cdot 15 + 1$

$5 = 8 \cdot 1 + 7$

$1 = 8 \cdot 0 + 1$

$(62290)_{10} = (171522)_8$\\


$62290 = 16 \cdot 3893 + 2$

$3893 = 16 \cdot 243 + 5$

$243 = 16 \cdot 15 + 3$

$15 = 16 \cdot 0 + 15$

$(62290)_{10} = (F352)_{16}$

\subsection*{2.4.}
(a)

$a = bq + r$

$2574 = 1976 \cdot 1 + 598$

$1976 = 598 \cdot 3 + 182$

$598 = 182 \cdot 3 + 52$

$182 = 52 \cdot 3 + 26$

$52 = 26 \cdot 2 + 0$

$gcd(2574,1976) = 26$

\noindent (b)

$1525 = 4405 \cdot 0 + 1525$

$4405 = 1525 \cdot 2 + 1355$

$1525 = 1355 \cdot 1 + 170$

$1355 = 170 \cdot 7 + 165$

$170 = 165 \cdot 1 + 5$

$165 = 5 \cdot 33 + 0$

$gcd(1525,4405) = 5$

$lcm(1525, 4405) = \frac{1525 \cdot 4405}{5} = 1343525$

\newpage
\section*{Exercise 3}

\noindent Assumptions

A1: $x,y,z,m \in \mathbb{Z}$

A2: $x \equiv y (\mathbf{mod}\ m)$

A3: $m|z$

\noindent Theorems

T1: $a|b \land a|c \to a | (b+c)$

T2: $a \equiv b (\mathbf{mod}\ m) \iff m | a - b$

\noindent To prove $x+2z \equiv y (\mathbf{mod}\ m)$

\noindent Proof.

By T2, $\exists m$ such that $m|x-y \land m|z$

By T1, $m|z \land m|x-y$ implies $m|(x-y)+z$ which you can rearrange to $m | (x+z)-y$

By T2, $m | (x-y) - z$ implies $x+z \equiv y (\mathbf{mod}\ m)$, where $a$ from T2 corresponds to $(x+z)$ and $b$ from T2 corresponds to $y$

By T1, $m|z \land m|z$ implies $m | z+z$ which means that adding another z wont affect the truth of the statement, therefore

$x+z \equiv y (\mathbf{mod}\ m)$ implies $x+2z \equiv y (\mathbf{mod}\ m)$

\QEDA

\end{document}
