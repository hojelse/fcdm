\documentclass[a4paper,11pt]{article}

% Import packages
\usepackage[a4paper]{geometry}
\usepackage[utf8]{inputenc}
\usepackage{amsmath}
\usepackage{amssymb}
\usepackage{bussproofs}
\usepackage{proof}
\usepackage{tikz}
\newcommand*{\QEDA}{\hfill\ensuremath{\blacksquare}}%
\newcommand*{\QEDB}{\hfill\ensuremath{\square}}%

\usepackage{listings}
\usepackage{color}

\definecolor{dkgreen}{rgb}{0,0.6,0}
\definecolor{gray}{rgb}{0.5,0.5,0.5}
\definecolor{mauve}{rgb}{0.58,0,0.82}

\lstset{frame=tb,
  language=Java,
  aboveskip=3mm,
  belowskip=3mm,
  showstringspaces=false,
  columns=flexible,
  basicstyle={\small\ttfamily},
  numbers=none,
  numberstyle=\tiny\color{gray},
  keywordstyle=\color{blue},
  commentstyle=\color{dkgreen},
  stringstyle=\color{mauve},
  breaklines=true,
  breakatwhitespace=true,
  tabsize=3
}

% Change enumerate environments you use letters
\renewcommand{\theenumi}{\alph{enumi}}

\title{Peergrade assignment 4}
\author{Author: Kristoffer Højelse} 
\date{\today}

\begin{document} 

\maketitle

\section*{Exercise 1}

Base case n = 7

$3^7 < 7! \iff 2187 < 5040$

Base case holds.\\

\noindent Assume $P(k)$ to prove $P(k+1)$

I.H.: $P(k) := 3^k < k!$

$P(k+1) := 3^{k+1} < (k+1)!$\\

\noindent $3^{k+1} = 3^k \cdot 3$ by factorization

by I.H.

$3^{k+1} < k! \cdot 3$

$3^{k+1} < k! \cdot (k+1)$, since $k>6$

$3^{k+1} < (k+1)!$

\QEDB

\section*{Exercise 2}

${f_0}^2 + {f_1}^2 + {f_2}^2 + \cdot\cdot\cdot + {f_n}^2 = f_n \cdot f_{n+1}$

$f(n) = \{0,1,1,2,3,5,8,...\}$

${f_0}^2 = f_0 \cdot f_{n+1} \iff 0 = 0 \cdot 1$

Base case holds.\\

\noindent Definition of Fibonacci sequence $f_n = f_{n-1} + f_{n-2}$\\

\noindent Assume $P(k)$ to prove $P(k+1)$

I.H.: $P(k) := \sum_{i=0}^k {f_i}^2 = f_k \cdot f_{k+1}$

$P(k+1) := \sum_{i=0}^{k+1} {f_i}^2 = f_{k+1} \cdot f_{k+2}$

$\iff \sum_{i=0}^k {f_i}^2 + {f_{k+1}}^2 = f_{k+1} \cdot f_{k+2}$

by I.H.

$\iff f_k \cdot f_{k+1} + {f_{k+1}}^2 = f_{k+1} \cdot f_{k+2}$

$\iff f_k \cdot f_{k+1} + {f_{k+1}}^2 = f_{k+1} \cdot (f_{k} + f_{k+1})$ by definition of the Fibonacci sequence.

$\iff f_k \cdot f_{k+1} + {f_{k+1}}^2 = f_{k} \cdot f_{k+1} + {f_{k+1}}^2$

\QEDB

\section*{Exercise 3}

The recursive algorithm is probably more efficient for n = 3 (maybe also for 4 and 5), and the integrative algorithm is more efficient for larger numbers.

The execution time of the iterative algorithm will grow about linearly as n gets bigger.

The execution time of the recursive algorithm probably grows with some exponential function.\\

\noindent Recursive algorithm

$a: n \to a(n)$

$a(0) := 1$

$a(1) := 3$

$a(2) := 5$

$a(n) = a(n-1) \cdot (a(n-2))^2 \cdot (a(n-3))^3$\\

\noindent Iterative algorithm

\begin{lstlisting}
int a(int n){
    int tmp0 = 1;
    int tmp1 = 3;
    int tmp2 = 5;
    
    if(i < 3){
        return 2*n+1;
    }
    
    if(i >= 3){
        int result;
        for(int i = 0; i < n+1; i++){
            result = tmp2 * tmp1*tmp1 * tmp0*tmp0*tmp0;
            tmp0 = tmp1;
            tmp1 = tmp2;
            tmp2 = result;
        }
        return result;
    }
    
}

\end{lstlisting}

\end{document}
