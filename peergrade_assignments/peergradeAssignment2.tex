\documentclass[a4paper,11pt]{article}

% Import packages
\usepackage[a4paper]{geometry}
\usepackage[utf8]{inputenc}
\usepackage{amsmath}
\usepackage{amssymb}
\usepackage{bussproofs}
\usepackage{proof}
\usepackage{tikz}
% \setlength{\parindent}{0em}

% Change enumerate environments you use letters
\renewcommand{\theenumi}{\alph{enumi}}

\title{Peergrade assignment 2}
\author{Author: Anon} 
\date{\today}

\begin{document} 

\maketitle

\section*{Exercise 1}

\subsection*{1.1. Which of the relations holds?}

\noindent (a) $A - B = \{4\}$ \textbf{false}

$A-B = \{4\} \to 4 \in A$ but $4 \notin A$ therefore the statement is false.\\

\noindent (b) $A \subseteq B \cup C$ \textbf{false}

Counterexample.

$5 \in A \land 5 \notin B \to A \nsubseteq B \to A \nsubseteq B \cap C$

Therefore the statement is false.\\

\noindent (c) $A - B \subseteq C$ \textbf{true}

$A-B = \{5\}$. Observe definition of C. If $x=5 \to 5 = 2k + 1 \iff k = 2$,

which means that $x=5 \to k \in \mathbb{N}$. C contains other elements different from 5.

Example: 69. $k=34 \to k \in \mathbb{N} \to x=2 \cdot k + 1 \to x = 69$.

Therefore the statement is true.\\

\noindent (d) $(A \times B) \cup (A \times C) \subset A \times \mathbb{N}$ \textbf{true}

Definition of C implies that $\forall x \in C, x \in \mathbb{N}$. Therefore $C \subseteq \mathbb{N}$.

Lastly $\exists x \in \mathbb{N}, x \notin C$, example: $6 \notin \mathbb{N}$ but $6 \notin C$. Therefore $C \subset \mathbb{N}$.

$C \subset \mathbb{N} \to (A \times C) \subset A \times \mathbb{N} \to (A \times B) \cup (A \times C) \subset A \times \mathbb{N}$

Therefore the statement is true.

\newpage
\subsection*{1.2. Is $f \circ g$ injective, surjective and or bijective?}

$f \circ g$:
\begin{center}
\begin{tikzpicture}[xscale=2,yscale = 1]
    \node (A) at (0,5) {$A$};
    \node (C) at (1,5) {$C$};
    \node (a) at (0,4) {$a$};
    \node (b) at (0,3) {$b$};
    \node (c) at (0,2) {$c$};
    \node (d) at (0,1) {$d$};
    \node (6) at (1,4) {$6$};
    \node (7) at (1,3) {$7$};
    \node (8) at (1,2) {$8$};
    \node (9) at (1,1) {$9$};
    \node (0) at (1,0) {$0$};
    \draw (a) -- (6);
    \draw (b) -- (8);
    \draw (c) -- (9);
    \draw (d) -- (0);
\end{tikzpicture}
\end{center}

\noindent (a) one-to-one (injective) \textbf{true}

$\forall x_1, x_2 \in X, F(x_1) = F(x_2) \to x_1 = x_2$

For every image there is a unique corresponding preimage.\\

\noindent (b) onto (surjective) \textbf{false}

$\forall y \in Y, \exists x \in X (F(x) = y)$

Counterexample

$\neg \exists x \in A (F(a)=7)$

There doesn't exist a pre-image for the image 7\\

\noindent (c) one-to-one correspondence (bijective) \textbf{false}

For $f \circ g$ to be bijective, $f \circ g$ has to be both injective and surjective,

but $f \circ g$ is not surjective therefore $f \circ g$ is not bijective.

\newpage
\subsection*{1.3. Hasse diagram}


\textbf{Graph a}

\noindent Partial orded set?

reflexive \textbf{true}

anti-symmetric \textbf{false}

transitive \textbf{true}

$\to$ partial orded set (poset) \textbf{false}\\

\noindent Equivalence relation?

reflexive \textbf{true}

symmetric \textbf{true}

transitive \textbf{true}

$\to$ equivalence relation \textbf{true}


$$[a]=\{x \in R | x R a\} = \{a,c,e\}$$

$$[b]=\{x \in R | x R b\} = \{b,d\}$$

$[a]=[c]=[e]$

$[b]=[d]$\\

\noindent \textbf{Graph b}

\noindent Partial orded set?

reflexive \textbf{true}

anti-symmetric \textbf{true}

transitive \textbf{true}

$\to$ partial orded set (poset) \textbf{true}\\

\noindent Equivalence relation?

reflexive \textbf{true}

symmetric \textbf{false}

transitive \textbf{true}

$\to$ equivalence relation \textbf{false}

\begin{center}
\begin{tikzpicture}[scale=1.5]
    \node (a) at (1,0) {$a$};
    \node (b) at (0,0) {$b$};
    \node (c) at (1,1) {$c$};
    \node (d) at (0,-1) {$d$};
    \node (e) at (1,2) {$e$};
    \node (f) at (0,1) {$f$};
    \draw (d) -- (b) -- (f);
    \draw (b) -- (c);
    \draw (a) -- (c) -- (e);
\end{tikzpicture}
\end{center}

Maximum: e, f

Minimum: d, a

Greatest: none

Least: none

\newpage
\section*{Exercise 2}
Provide a proof by contradiction of this statement: Let $f : Ø \to A$ be a function with domain the empty set and an arbitrary set A as co-domain. Then f is one-to-one (injective).\\

\noindent Proof.\\

Suppose that the negation of the definition of one-to-one is true,

$
\exists x_1, x_2 \in \emptyset (F(x_1) = F(x_2) \to x_1 \neq x_2)
$,

where $x_1$ and $x_2$ is a arbitrary element.

By definition of the empty set

$\forall x \{ x \notin \emptyset \}$

Therefore $x_1 \in \emptyset$ and $x_1 \notin \emptyset$, which is a contradiction.

Therefore $f : \emptyset \to A$ is a one-to-one function.

\section*{Exercise 3}
\textit{Provide a direct proof of this statement:} Let A and B be two particular but arbitrary
sets such that $B \subseteq A$, and let R be an equivalence relation on A. Consider the relation
$S = \{(a,b) \in R | a,b \in B \}$ on the set B. Then S is an equivalence relation on B (that
is, reflexive, symmetric and transitive).\\


\noindent Proof.\\

\noindent S is an equivalence relation on B if the set of relations describing relations between nodes in B has reflexive, symmetric and transitive properties.\\

\noindent Sub-proof that S is reflexive.


Definition of reflexive: R is reflexive $\iff \forall x \in A, x R x$.

If $a \in A \land b \in B$ such that $a=b$ it means that $(a,a) \in R$ so per definition of S, $S = \{(a,b) \in R | a,b \in B \}$, it means that $(a,a)$ must be in S.

Therefore $\forall x \in B, x R x$ which means that S is a reflexive relation on B.\\

\noindent Sub-proof that S is symmetric.

Definition of symmetric: R is symmetric $\iff \forall x, y \in A, x R y \to y R x $.

Per definition of S, $S = \{(a,b) \in R | a,b \in B \}$, it means that $(a,b) \in R \to (b,a) \in R$.

So if $a,b \in B \land (a,b) \in R \to ((a,b) \in R \to (b,a) \in R) \to (a,b),(b,a) \in S$.

Therefore $\forall a, b \in B, a R b \to b R a$ which means that S is a symmetric relation on B.\\

\noindent Sub-proof that S is transitive.

Definition of transitive: R is transitive $\iff \forall x, y, z \in A, x R y \land y R z \to x R z$

Let $a,b,c \in A$ and $a,b,c \in B$.

Since R is transitive, then if $(a,b) \in R$ and $(b,c) \in R$ then $(a,c) \in R$ by definition of transitive $\forall x, y, z \in A, x R y \land y R z \to x R z $. So if (a,b) is in S and (b,c) is in S then (a,c) must be in S, since $a,b,c \in B$ and $a,c \in R$.

Therefore $\forall a,b,c \in B, a R b \land b R c \to a R c$ which means that S is a transitive relation on B.\\

\noindent Conclusion.

S both a reflexive, symmetric and transitive relation on B, therefore S is an equivalence relation on B.


\end{document}
